% !TEX TS-program = pdflatex
% !TEX encoding = UTF-8 Unicode

% This is a simple template for a LaTeX document using the "article" class.
% See "book", "report", "letter" for other types of document.

\documentclass[11pt]{article} % use larger type; default would be 10pt

\usepackage[utf8]{inputenc} % set input encoding (not needed with XeLaTeX)

%%% Examples of Article customizations
% These packages are optional, depending whether you want the features they provide.
% See the LaTeX Companion or other references for full information.

%%% PAGE DIMENSIONS
\usepackage{geometry} % to change the page dimensions
\geometry{a4paper} % or letterpaper (US) or a5paper or....
% \geometry{margin=2in} % for example, change the margins to 2 inches all round
% \geometry{landscape} % set up the page for landscape
%   read geometry.pdf for detailed page layout information

\usepackage{graphicx} % support the \includegraphics command and options

% \usepackage[parfill]{parskip} % Activate to begin paragraphs with an empty line rather than an indent

%%% PACKAGES
\usepackage{booktabs} % for much better looking tables
\usepackage{array} % for better arrays (eg matrices) in maths
\usepackage{paralist} % very flexible & customisable lists (eg. enumerate/itemize, etc.)
\usepackage{verbatim} % adds environment for commenting out blocks of text & for better verbatim
\usepackage{subfig} % make it possible to include more than one captioned figure/table in a single float
% These packages are all incorporated in the memoir class to one degree or another...

%%% HEADERS & FOOTERS
\usepackage{fancyhdr} % This should be set AFTER setting up the page geometry
\pagestyle{fancy} % options: empty , plain , fancy
\renewcommand{\headrulewidth}{0pt} % customise the layout...
\lhead{}\chead{}\rhead{}
\lfoot{}\cfoot{\thepage}\rfoot{}

%%% SECTION TITLE APPEARANCE
\usepackage{sectsty}
\allsectionsfont{\sffamily\mdseries\upshape} % (See the fntguide.pdf for font help)
% (This matches ConTeXt defaults)

%%% ToC (table of contents) APPEARANCE
\usepackage[nottoc,notlof,notlot]{tocbibind} % Put the bibliography in the ToC
\usepackage[titles,subfigure]{tocloft} % Alter the style of the Table of Contents
\renewcommand{\cftsecfont}{\rmfamily\mdseries\upshape}
\renewcommand{\cftsecpagefont}{\rmfamily\mdseries\upshape} % No bold!

%%% END Article customizations

%%% The "real" document content comes below...

\title{Javascript Notes}

\begin{document}
\maketitle

\tableofcontents

\section{Scope}

\paragraph{} The global namespace is available as a global object. The \emph{this} keyword intitally
references this object.

\paragraph{Q:} Generally global variables are bad, so why would you need to use the global scope?\\
A: To interact between seperate program components. To detect features of the host
environment.

\paragraph{Q:} How can you accidently create a global variable within a function?\\
A: By declaring without the \emph{var} keyword

\paragraph{Q:} Scope chain vs. Prototype chain?\\
A: 

\section{Functions}

\subsection{General}

\paragraph{} A function is a function. A method is a function that is an object's property. Constructor
functions are invoked with \emph{new}.

Within a method \emph{this} is a reference to invoking object (\emph{receiver}), not the defining object. In non-method
functions \emph{this} refers to the global object.

\paragraph{Q:} What if I want to call function \emph{f} as a method of object \emph{o}, but \emph{f} is not a
method/property of \emph{o}?\\
A: Use \emph{call}, a method every function (including \emph{f}) posesses. You can pass in your desired
\emph{reciever} object along with arguments.

\paragraph{Q:} What if I want to extract a function's method and pass it to a higher order function?
A: Use bind to prevent the incorrect reciever prob

\subsection{Callbacks}

\paragraph{Overview} Function\emph{b} is passed into function \emph{a} as an argument and \emph{a}
executes \emph{b} at the end of its body. The main consideration in using callbacks is asynchronous-ness.


\section{Closures}

\paragraph{Overview} Functions that store references to variables from their
containing/enclosing scopes. State hidden within behavior.

\paragraph{} Closures can update variables from enclosing scopes. Inner functions contain the scope of parent functions even if the parent function has returned.

\section{Operators}

\paragraph{Q:} == vs === ?

\section{Libraries}

\subsection{underscore.js}

\paragraph{Overview} A bunch of functions to make javascript more functiontional - provides map, reduce,
filter, etc.

\subsection{require.js}

\paragraph{Overview} To write modular javascript while optimizing by reducing HTTP requests for js
resources.

\paragraph{Modules} Are scoped so they don't pollute the global namespace. A good description of the
Javascript module pattern at \emph{http://www.adequatelygood.com/JavaScript-Module-Pattern-In-Depth.html}

\paragraph{} Each module file is of the basic form define( STUFF ); Where STUFF parameters can include 
dependencies (other modules) and the definition of an object literal or function. RequireJS figures out all
dependencies


\subsection{prototype.js}

\subsection{backbone.js}

\paragraph{Overview} Client-side MVC.

\paragraph{Models} Data/State is represented as \emph{Models}. When a model gets changed it fires a \emph{change}
event. All Views that display that model's state update themselves in response to the change.

\paragraph{Collections} Ordered sets of models

\paragraph{Views} To represent models as DOM elements.




\end{document}
